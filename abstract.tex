Nonlinear optical processes provide a valuable technique for probing properties of chiral structures, those lacking mirror symmetry. 
Such structures are encountered frequently throughout organic chemistry, pharmacology, and biology, with a recent growing emphasis on sensitive characterisation of small quantities of chiral molecules. 
	    
It has been shown that chiroptical effects such as circular dichroism and optical rotation are significantly enhanced in second-harmonic generation. 
Additionally, plasmonic nanostructures can create local regions of high intensity, highly chiral electromagnetic fields that further enhance chiroptical interactions with molecules attached to the material surface. 
	    
In this report, chiroptical and nonlinear optical properties of dielectric materials are reviewed in the context of both natural media and plasmonic nanomaterials. 
Using these principles, three novel chiral analysis schemes are experimentally demonstrated. Each experiment is designed for the characterisation of a unique plasmonic nanomaterial, all making use of enhanced nonlinear chiroptical effects. 
By designing new experimental schemes considering the unique challenges associated with each nanomaterial geometry, previously unobserved optical properties of the materials are revealed. Such tailored experimental methods pave the way for the further optimisation of chiral nanomaterials designed for nanorobotic, photonic devices, and chemical sensing applications.
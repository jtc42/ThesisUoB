Nonlinear optical processes provide a valuable technique for probing properties of chiral structures, those lacking mirror symmetry. 
Such structures are encountered frequently throughout organic chemistry, pharmacology, and biology, with a recent growing emphasis on sensitive characterisation of small quantities of chiral molecules. 
	    
It has been shown that chiroptical effects such as circular dichroism are significantly enhanced in second-harmonic generation. 
Additionally, plasmonic nanostructures can create local regions of high intensity, highly chiral electromagnetic fields that further enhance chiroptical interactions with molecules doped onto the material surface. 
	    
In this report, chiroptical and nonlinear optical properties of dielectric materials are reviewed in the context of both natural molecules and plasmonic nanostructures. 
Current work is outlined exploring the modal composition of strong chiroptical responses in plasmonic nanostructures. 
Simulations relating to an ongoing experiment are then detailed, showing how diffraction can reveal additional information about a the chirality of a periodic array of nanostructures. Progress on a collaborative project studying chiral molecular aggregates is also outlined. Future work will look to develop comprehensive nonlinear characterisation of deep sub-wavelength scale chiral metamaterials.
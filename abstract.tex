Nonlinear optical processes provide a valuable technique for probing properties of chiral structures, those lacking mirror symmetry. 
Such structures are encountered frequently throughout organic chemistry, pharmacology, and biology, with a recent growing emphasis on sensitive characterisation of small quantities of chiral molecules. 
	    
It has been shown that chiroptical effects such as circular dichroism and optical rotation are significantly enhanced in second-harmonic generation. 
Additionally, plasmonic nanostructures can create local regions of high intensity, highly chiral electromagnetic fields that further enhance chiroptical interactions with molecules attached to the material surface. 
	    
In this report, chiroptical and nonlinear optical properties of dielectric materials are reviewed in the context of both natural media and plasmonic nanomaterials. 
Using these principles, three novel chiral analysis schemes are experimentally demonstrated. Each experiment is designed for the characterisation of a unique plasmonic nanomaterial, all making use of enhanced nonlinear chiroptical effects. 

We first consider plasmonic structures with dimensions comparable to the wavelength of incident light. In these structures, electric field hotspots form, leading to small regions of enhanced chiral and nonlinear optical interactions. A theoretical description making use of a modal analysis and group theory is found to well match experimental results. Following this, we consider the effects of anisotropy on nonlinear chiroptical measurements of planar nanomaterials. The structures are of sub-wavelength dimensions, however exhibit strong anisotropy that can contribute to chiroptical effects. We demonstrate that specific experimental configurations can be exploited to separate the contributions from structural chirality and anisotropy, allowing pure chiral information to be obtained from highly anisotropic materials. Finally, we consider the case of removing structural anisotropy by dispersing anisotropic nanostructured in liquid. In this case, nonlinear scattering is measured, and found to exhibit strong chiroptical effects. This is the first experimental report of second-harmonic scattering circular intensity difference, and demonstrates significantly enhanced sensitivity when compared to comparable, widely used, linear characterisation techniques.

By designing new experimental schemes considering the unique challenges associated with each nanomaterial geometry, previously unobserved optical properties of the materials are revealed. Such tailored experimental methods pave the way for the further optimisation of chiral nanomaterials designed for nanorobotic, photonic devices, and chemical sensing applications.
\chapter{Summary and Conclusions}\label{sec:Conclusions}

\color{red}
Key point for each chapter:
\begin{itemize}
    \item Probes for chirality are incredibly important, and light is a great tool to characterise chiral systems.
    \item Nonlinear optics enhance many chiroptical measurements, due to the extreme symmetry sensitivity and intrinsic achiral background removal of SHG.
    \item Plasmonic nanomaterials can be used to enhance both the optical chirality of light at the material surface, and the field intensity. This leads to enhanced SHG, and the combined effect of this, and enhanced optical chirality has made plasmonic nanomaterials a leading platform for chiral sensing and characterisation experiments.
    \item We demonstrated that the flexibility of nanomaterial fabrication allows chirality to be studied in entirely new ways, not previously allowed by examining chiral molecular systems. 
    \begin{itemize}
        \item By enantiomorphing a chiral structure from one handedness to the other, through intermediate geometries, we observed a counter-intuitive double-bisignate SHG-CID. 
        \item A theoretical analysis revealed a model for this behaviour based on the coupling of circularly polarised light to current-density modes at the structure surface.
        \item Both the experimental results and the theoretical model demonstrated the ability to tune the chiroptical properties of a nanostructure by changing simple geometric parameters, with predictable behaviour from modal analysis of the geometry. At any particular, fixed incident wavelength, this tuneability can be used to increase enantioselectivity.
    \end{itemize}
    \item By next considering a metamaterial, consisting of an array of sub-wavelength plasmonic inclusions, we were able to demonstrate the ability to use the flexible tuneability of plasmonic nanostructures to disentangle the contributions of anisotropy and chirality to SHG optical rotation.
    \begin{itemize}
        \item Previous work~\cite{Hooper2017} has shown that nonlinear chiroptical effects in metasurfaces are highly sensitive to structural anisotropy. In many molecular systems, this is less of a concern since the chiral molecules are often isotropically dispersed in solution. In artificial metasurfaces however, this is rarely the case, and anisotropy can significantly complicate chiroptical characterisation.
        \item We show that under specific experimental and sample geometries, a ``Goldilocks'' condition can be met in which anisotropic contributions to SHG-OR cancel out at all sample rotation angles. 
        \item While the intensity of SHG emission is still strongly related to the anisotropy of the structure, the polarisation of SHG emission is affected exclusively by the structures intrinsic chirality.
    \end{itemize}
    \item Finally, we demonstrate an alternative method to extract pure chiral information from plasmonic nanostructures: By suspending the nanohelices in water, we measured nonlinear scattering (HRS) from the, now isotropically arranged, nanomaterial.
    \begin{itemize}
        \item The nonlinear scattering showed clear circular intensity difference, reversing near perfectly between helix enantiomorphs.
        \item Crucially, the concentration of suspended nanohelices was on the order of picomolars, and especially when considering the volume of interaction, the effect is shown to be orders of magnitude stronger than linear CD measurements.
        \item Pure chiral information is directly obtained, allowing the potential for HRS-CID to be leveraged as a sensitive characterisation technique for trace quantities of chiral structures or molecules.
    \end{itemize}
    
    \item All of this work has focused on the development of novel characterisation techniques for true chirality, utilising nanotechnology. The design flexibility of nanomaterials has allowed us to fully utilise structural tuneability to explore previously unobserved chiral behaviour, disentangle chirality from anisotropy, and detect pure chirality in low concentration nanomaterial liquids.

    \item Discuss contribution in the broader context of the field. [``The key to many dissertations and theses is the need to emphasise the contribution that it makes to research.'']

    \item It is my hope that in the future, the three techniques discussed will be fully utilised in new applications... [some kind of closing statement...]

\end{itemize}
\color{black}
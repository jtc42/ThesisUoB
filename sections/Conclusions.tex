\chapter{Summary and Conclusions}\label{sec:Conclusions}

With the availability of modern nanofabrication techniques, chiral plasmonic nanomaterials, in particular, have opened up the field of nanophotonics to a wide range of novel devices. By optimising structural chirality parameters, negative index metamaterials have been realised, with applications in sub-wavelength imaging. Similar effects have also been utilised for ultra-thin polarisation optics. Perhaps more surprisingly, chiral plasmonic devices have also introduced a new platform for nanorobotics. Key to perhaps more humanitarian applications however is the possibility to realise new, sensitive chemical characterisation techniques for the pharmaceutical and biochemistry industries. 

Development within these research areas, and more, has been rapidly accelerated by advances in nanofabrication, and the unparalleled freedom that comes with the use of nanomaterials. The fabrication of novel nanomaterial geometries directly lifts limits on the optical effects available for exploration. In many cases however, the diversity of characterisation experiments has not yet matched the diversity of material geometries available: many ``off-the-shelf'' experimental techniques are not necessarily perfectly suited to the characterisation of particular nanomaterial geometries. This work has focused on developing experimental techniques designed for the characterisation of chiral nanomaterials fabricated by our collaborators, by specifically considering the unique properties of each, and the experimental challenges these properties may present. 

Common to each of the samples studied is the comprehensive use of second-harmonic nonlinear optical effects as probes for chirality. Second-harmonic chiroptical effects have been shown to function as highly sensitive probes for chirality, due to the symmetry sensitivity of second-harmonic generation. In experiments characterising chiral structures within an achiral background, SHG is generally forbidden from the achiral bulk, thus providing intrinsic background reduction. Furthermore, and contrary to linear optical counterparts, SHG chiroptical effects are permitted within the electric dipole approximation, providing entirely new information due to their fundamentally different physical origin. By utilising these various enhancements, we have demonstrated a set of analysis techniques for three very different chiral plasmonic nanomaterials.

Plasmonic structure arrays with dimensions comparable to the wavelength of light can support higher-order plasmon modes, whose spatial profiles directly relate to the material geometry. We have shown that by examining intermediate geometries between opposite enantiomorphs, unexpected optical behaviour can be observed. Electric field hotspots form at the centre of our chiral cross structures, originating from higher-order hybridised plasmon modes of the coupled nanostripes composing the crosses. Structural chirality directly leads to a circular difference in the hotspot intensity, which was measured using SHG microscopy. By considering modal decompositions of the total plasmonic responses, new avenues for chiroptical optimisation are opened. By tuning the structure geometry, or introducing auxiliary processes that modify the coupling strengths of specific modes, the chiroptical response can be enhanced by suppressing contributions from the ``wrong'' modes.

In many such planar nanomaterials, structural anisotropy can ``dilute'' chiroptical measurements, making it difficult to extract pure chiral information over contributions from anisotropy. Previous work has shown that in a helical metamaterial, nonlinear circular intensity difference effects are dominated by structural anisotropy. By instead measuring SHG optical rotation from same helical metamaterial, we have shown that, under specific experimental conditions, it is possible to obtain a chiroptical response that provides separable information about chirality and anisotropy. The conditions depend strongly on both the nanomaterial geometry, and the illumination conditions. Changing, for instance, the polarisation of incident light resulted in a response completely dominated by anisotropy, with no separable chiral information available. Nevertheless, the work demonstrated, for the first time, SHG optical rotation from a chiral metamaterial, which can be exclusively attributed to intrinsic chirality. Carefully selecting the experimental geometry for a particular sample geometry can generally allow for more comprehensive chiral characterisation of anisotropic metamaterials in the future.

Finally, we examined the case of intrinsically removing anisotropy from the experimental system, by suspending randomly oriented silver nanohelices in liquid. In our previous experiments, nonlinear chiroptical effects could be measured by detecting SHG in a reflection geometry. For liquid samples, we employed a hyper-Rayleigh scattering (HRS) geometry, to measure the circular difference in second-harmonic scattering of incident light. Strong HRS circular intensity difference was observed at three different incident wavelengths, well above background. Within the electric dipole approximation, HRS-CID is symmetry forbidden. Our observations can be explained by both magnetic dipole contributions to HRS, and the helices behaving as helical arrangements of coupled dipoles, rather than as point scatterers themselves. Future experiments outside of the scope of this thesis may be able to determine the dominant of these contributions to the observed HRS-CID. In this work we have nevertheless reported the first experimental observation of HRS-CID in a liquid sample, providing pure chiral information independent of the individual nanohelices anisotropy.

As our understanding of chirality in nanophotonics develops, and nanofabrication techniques open up even more design flexibility, it should be expected that the need for tailored characterisation experiments will blossom. It seems only reasonable that the freedom and creativity permitted by modern nanofabrication should be matched by novel detection and characterisation schemes. Throughout this project, we have demonstrated a range of techniques designed to reveal previously unobserved behaviour from particular nanomaterial samples. By more fully exploiting the level of freedom permitted by such systems, the field of chiral nanophotonics will no doubt benefit from specially tailored characterisation techniques in the future, allowing for more rapid design, characterisation, and enhancement of chiral optical nanomaterials.

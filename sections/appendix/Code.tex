Significant time during this project has been spent developing Python libraries to automate the acquisition and analysis of data taken from a range of experimental setups. This section briefly discusses the purpose of the most significant libraries developed, and where to find the publicly available source code.

\section{Automated Data Acquisition}\label{sec:appendix:labdo}
Diffraction CID experiments undertaken by C. Kuppe (reference~\cite{Kuppe2018}) made use of automated lab hardware from a range of manufacturers, each using their own software platforms for automation. Moreover, we found early on that errors in one piece of hardware could result in difficult to recover errors in other parts of the experimental setup. With this in mind, a Python library was developed to alleviate these two main issues. 

The library defined a common class for all lab hardware, requiring certain functions like standard and emergency shutdown procedures. When opening communication with a device, the system attaches that device to an object responsible for tracking the status of all active devices. Communication cannot be opened to a device without this. In the event of a repeated, irrecoverable exception in communication, all devices are automatically shut down in the safest way possible. Additionally, when the experiment is finished, communication with all devices is automatically safely closed, without any required user input. All behaviour can, however, be overridden by the lab user if needed. 
Parts of this library were used in the acquisition of chiroptical microscopy measurements from section~\ref{sec:results:OAinPlanarNanohelices}, and may be used in SHG measurements in the future.

The developed code was recently merged into a fork of the ``LabDo'' module, by J. Stirling. The fork can be found at \url{https://gitlab.com/jtc42/LabDo}.
The specific implementation of this library used by C. Kuppe in all diffraction CID measurements can be found at \url{https://gitlab.com/jtc42/LabDo-CD-Spectrometer}.

\section{Automated Data Analysis}\label{sec:appendix:pyshg}
Due to the large number of degrees of freedom during SHG chiroptical measurements, as found in references~\cite{Collins2018b, Hooper2017}, any individual heatmap required the processing and analysis of many thousands of data points. For this reason, another Python 3 library was developed to automate this analysis as much as reasonably possible. By using metadata embedded by LabView acquisition code developed by D. C. Hooper, all previously unsorted and unfiltered data can be pulled into the library, and plots for common typed of experiment can be automatically generated. This included heatmaps for SHG-CID measurements, as well as SHG optical rotation heatmaps (and curve fitting data) such as those in section~\ref{sec:results:OAinPlanarNanohelices}.
The developed code can be found at \url{https://github.com/jtc42/mpnp-shg-analysis}.
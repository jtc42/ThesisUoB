\chapter{Structure and Overview}\label{sec:background:Introduction}
\color{red}
This chapter might be unnecessary, as the motivation is kind of discussed in the Chirality chapter. A brief overview of the chapter structure might be nice though?
Mention that bits of the background sections are lifted from my parts of the review paper! Mention microscopy work with Dave Carbery.

\begin{itemize}
    \item Understanding chirality is important
    \item Chemicals have been used to understand chirality
    \item However, nanomaterials can be better (ref section)
\end{itemize}
\color{black}

Time has also been spent developing Python libraries to simplify data acquisition and analysis on a range of experimental setups within the group. While not discussed in this thesis, diffraction CID experiments undertaken by C. Kuppe (reference~\cite{Kuppe2018}) made use of an experiment-automation library I developed during this project, and eventually merged into a larger library, detailed in appendix~\ref{sec:appendix:labdo}. Data analysis shown in section~\ref{sec:results:OAinPlanarNanohelices} made use of another Python library I developed to automate the filtering, processing, and plotting of data from our SHG optical activity setup. The library, detailed in appendix~\ref{sec:appendix:pyshg}, has since been used in submitted work by D. C. Hooper, and is expected to be used in similar experiments going forward.
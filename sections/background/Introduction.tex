\chapter{Structure and Overview}\label{sec:background:Introduction}

This thesis reports three new experimental schemes developed to provide nonlinear chiral-optical characterisation of plasmonic nanomaterial samples fabricated by our collaborators. 
The diversity of available nanostructure geometries has allowed us to develop new, tailored characterisation and analysis schemes. Previously unobserved novel optical behaviour is reported in each case.

Chapter~\ref{sec:results:EnantiomorphingChiralCrosses} examines nanostructures with dimensions comparable to the wavelength of light, as the structure is ``enantiomorphed'' from one handedness to the other. Far-field nonlinear microscopy is used to probe the chiral properties of the nanostructures' local electromagnetic fields. Making use of group theory reveals the modal origin of strong experimentally observed near-field circular intensity difference. 

A dramatically different nonlinear characterisation technique is presented in chapter~\ref{sec:results:OAinPlanarNanohelices}. Second-harmonic generation optical rotation is used to probe a metamaterial consisting of sub-wavelength gold nanohelices. The individual structures are highly anisotropic, as well as chiral, and separating the effects of anisotropy and chirality is challenging. This chapter details the use of specific experimental configurations, and sample geometries, to disentangle the effects of anisotropy and true chirality. 

Finally, chapter~\ref{sec:results:HRS} discusses the first ever experimental observation of circular intensity difference in hyper-Rayleigh scattering, from an isotropic suspension of silver nanohelices. Measuring this effect is found to provide pure chiral information, free of significant contributions from structural anisotropy. In simple models of hyper-Rayleigh scattering from point-like scatterers, this chiroptical effect is symmetry-forbidden, and a physical origin is proposed by considering higher-order contributions based on existing models of molecular chirality.

Across all of these experiments, significant time has been spent developing Python libraries to simplify data acquisition and analysis within our research group. Chiral diffraction experiments undertaken by C. Kuppe (reference~\cite{Kuppe2018}) made use of an experiment-automation library I developed during this project, and eventually merged into a larger library, detailed in appendix~\ref{sec:appendix:labdo}. Data analysis shown in chapters~\ref{sec:results:OAinPlanarNanohelices} and~\ref{sec:results:HRS} made use of another Python library I developed to automate the filtering, processing, and plotting of data from our SHG optical activity setup. The library, detailed in appendix~\ref{sec:appendix:pyshg}, has since been used in submitted work by D. C. Hooper, and is expected to be used in similar experiments going forward.

We will first begin in chapter~\ref{sec:background:Chirality} by discussing the general prevalence of chirality in nature, and the use of photonics as a platform for sensitive chiral characterisation techniques. Nonlinear optical effects in particular provide symmetry sensitivity that can be used to enhance chiral optical measurements, and are discussed in chapter~\ref{sec:background:NonlinearOptics}. Specific to this work is the use of plasmonic nanomaterials to provide flexible, enhanced chiroptical \textit{and} nonlinear optical interaction, and will be discussed in chapter~\ref{sec:background:Plasmonics}. 




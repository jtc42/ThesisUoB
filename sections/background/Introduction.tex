\chapter{Structure and Overview}\label{sec:background:Introduction}
\color{red}

\begin{itemize}
    \item Introduce somehow?
\end{itemize}
\color{black}

Outside of the experiments reported in this thesis, time has also been spent developing Python libraries to simplify data acquisition and analysis on a range of experimental setups within the group. While not discussed in this thesis, diffraction CID experiments undertaken by C. Kuppe (reference~\cite{Kuppe2018}) made use of an experiment-automation library I developed during this project, and eventually merged into a larger library, detailed in appendix~\ref{sec:appendix:labdo}. Data analysis shown in section~\ref{sec:results:OAinPlanarNanohelices} made use of another Python library I developed to automate the filtering, processing, and plotting of data from our SHG optical activity setup. The library, detailed in appendix~\ref{sec:appendix:pyshg}, has since been used in submitted work by D. C. Hooper, and is expected to be used in similar experiments going forward.

We will begin in chapter~\ref{sec:background:Chirality} by discussing the prevelance of chirality in nature, the need for sensitive chiral characterisation techniques, and the use of photonics as a platform for this. Nonlinear optical effects in particular provide symmetry sensitivity that can be used to enhance chiral optical measurements, discussed in chapter~\ref{sec:background:NonlinearOptics}. Specific to this work is the use of plasmonic nanomaterials to provide flexible, enhanced chiroptical \textit{and} nonlinear optical interaction, as discussed in chapter~\ref{sec:background:Plasmonics}. The following chapters report three new experimental schemes developed to provide comprehensive chiroptical charactersation of unique plasmonc nanomaterial samples fabricated by our collaborators. Each sample exhibits very different structural properties, and so new, tailored characterisation and analysis schemes have been developed, with previously unobserved novel optical behaviour reported in each case.
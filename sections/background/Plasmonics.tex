\chapter{Plasmonic Nanomaterials}\label{sec:background:Plasmonics}

\section{Introduction}
\begin{itemize}
    \item Very brief overview of propagating surface plasmon polaritons
    \begin{itemize}
        \item Dispersion relations of SPP and light, momentum matching conditions
        \item Conservation of momentum applies. Effective mass of the SPP and the fact it propagates necessitates momentum matching in order to resonantly drive the SPP with light.
        \item Very brief description of the experimental geometries used for momentum matching (Kretschmann and Otto geometries)
    \end{itemize}
    \item Properties and applications of propagating plasmons, ie SERS
\end{itemize}

\section{Localised Surface Plasmons}\label{sec:background:Plasmonics:Metamaterials}
\begin{itemize}
    \item Quasi-static approximation: field is constant over the size of the particle
    \item Sub-wavelength nanoparticles, removing need for momentum matching
    \item LSP resonances determined by the nanoparticle material, shape, and dimensions in the dipole approximation (valid for ``vanishingly small'', deep sub-wavelength, particles.)
    \begin{itemize}
        \item Example for spherical nanoparticles (well established)
        \item Use this example to qualitatively discuss how the particle acts a bit like a Lorentz oscillator
        \item In many systems the nanoparticle geometry is significantly more complex, and the LSP resonance cannot be determined analytically
    \end{itemize}
    \item Applications in photonic
    \begin{itemize}
        \item Sub-wavelength field enhancement/confinement between coupled nanoparticles, and super-resolution imaging
        \item Introduce concept of metamaterials/effective media
    \end{itemize}
\end{itemize}

\section{Quasi-Localised Surface Plasmons and Plasmonic Modes}
\begin{itemize}
    \item Intermediate regieme of nanostructures, beyond the quasi-static approximation. Structure dimensions are close to, or larger than, the wavelength.
    \item The near-field optical behaviour in this regieme can be understood in terms of current density modes. 
    \begin{itemize}
        \item The geometry of the metallic surface will permit a set of plasmon modes
        \item Higher order modes can be excited across the surface of the structure. 
        \item The plasmonic modes are effectively non-propagating current-density standing waves, and thus have zero net momentum. Thus, even in this regieme momentum matching is not a requirement to achieve plasmon resonance.
    \end{itemize}
    \item Zheng's work demonstrated that the plasmonic response of metallic nanostructures in this intermediate regieme can be decomposed into a set of current density eigenmodes~\cite{Zheng2012}. This work was extended to collections of coupled nanostructures~\cite{Zheng2013}. 
    \item The optical response can be further understood by applying group theory to the plasmonic eigenmode set~\cite{Zheng2015}, allowing the current density modes to be grouped by the optical polarisation states that exclusively excite them.
\end{itemize}

\section{Chiral Plasmonic Nanostructures}
\begin{itemize}
    \item Literature review
\end{itemize}

\subsection{Superchiral Fields from Plasmonic Metasurfaces}
\begin{itemize}
    \item Literature review
\end{itemize}
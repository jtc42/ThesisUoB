\section{Enantiomorphing chiral crosses}\label{sec:results:EnantiomorphingChiralCrosses}
\textit{This section has largely been copied verbatum from the manuscript ``Enantiomorphing Chiral Plasmonic Nanostructures: A Counterintuitive Sign Reversal of the Nonlinear Circular Dichroism'', Joel~T.~Collins et. al. \cite{Collins2018}.} \\

Plasmonic nanostructures have demonstrated a remarkable ability to control light in ways never observed in nature, as the optical response is closely linked to their flexible geometric design.
Due to lack of mirror symmetry, chiral nanostructures allow twisted electric field “hotspots” to form at the material surface. 
These hotspots depend strongly on the optical wavelength and nanostructure geometry.
Understanding the properties of these chiral hotspots is crucial for their applications; for instance, in enhancing the optical interactions with chiral molecules. 
Here, the results of an elegant experiment are presented: by designing 35 intermediate geometries, the structure is “enantiomorphed” from one handedness to the other, passing through an achiral geometry. 
Nonlinear multiphoton microscopy is used to demonstrate a new kind of double-bisignate circular dichroism due to enantiomorphing, rather than wavelength change.
From group theory, a fundamental origin of this plasmonic chiroptical response is proposed. The analysis allows the optimization of plasmonic chiroptical materials.

\subsection{Introduction}\label{sec:results:EnantiomorphingChiralCrosses:introduction}
Throughout the 19th and most of the 20th century, chirality has been associated with chemistry. However, whereas chirality can be crucial for understanding molecules, molecules are not best suited for understanding chirality. Indeed, there are various forms of chirality, such as helical chirality, propeller chirality, supramolecular chirality, extrinsic chirality, etc.\cite{Collins2017, Valev2013b}. These forms all depend on chirality parameters. 
Ideally, we would like to be able to vary these parameters, i.e., to follow the parameter values as chiral systems transition from one chiral form into another. However, it is impossible to control the size of atoms, the length of chemical bonds, and the orientation of orbitals. Modern nanofabrication techniques have lifted the limitations on tuning chiral parameters.

Using modern nanofabrication methods, it is possible to explore the evolution of chiral forms, by preparing numerous intermediate geometries. This is important because it opens the possibility to tune and optimize the chirality parameters, which enable interesting properties. For instance, by maximizing the geometric chirality parameter, it is possible to achieve negative refractive index materials \cite{Pendry2004a}. Such materials could lead to super-lenses \cite{Khorasaninejad2016}, and various applications that depend on the control of circularly polarized light (CPL). In turn, CPL could find applications in spintronics \cite{Farshchi2011b} and quantum computing \cite{Wagenknecht2010a, Sherson2006a}.
Moreover, by optimizing a parameter called \textit{optical chirality} \cite{Tang2010} it has been shown that “superchiral” light configurations can be achieved. In such configurations, the pitch of the electric field of light is shorter than that of circularly polarized light, thereby enabling stronger chiroptical effects \cite{Hendry2010, Hendry2012, Tullius2015}. Importantly, optical chirality is particularly enhanced at the surface of chiral plasmonic nanostructures \cite{Schaferling2012, Karimullah2015}, resulting in large enhancements in measurable circular dichroism (CD) \cite{Maoz2013, Wang2014c, Ma2013b, Zhang2013}.
\chapter{Optical Activity in Hyper-Rayleigh Scattering}\label{sec:results:HRS}

Sections of this chapter have been copied verbatim from the (submitted) manuscript \textit{``First observation of optical activity in hyper-Rayleigh scattering''}.
I designed the nonlinear experimental setup. H.-H. Jeong and P. Fischer fabricated the samples. HRS data were collected by myself, K. R. Rusimova, and D. C. Hooper. Linear optical data were collected by myself, K. R. Rusimova, and F. Pradaux-Caggiano. I analysed both the HRS and linear data. I produced the manuscript draft, and all authors subsequently contributed.

\bigskip \noindent
Chiral nano/metamaterials and surfaces enable striking photonic properties, such as negative refractive index and superchiral light, driving promising applications in novel optical components, nanorobotics, and enhanced chiral molecular interactions with light. In characterizing chirality, although nonlinear chiroptical techniques are typically much more sensitive than their linear optical counterparts, separating true chirality from anisotropy is a major challenge. Here, we report the first observation of optical activity in hyper-Rayleigh scattering (HRS). We demonstrate the effect in a 3D isotropic suspension of Ag nanohelices in water and show a complete sign reversal depending on handedness. The effect is 5 orders of magnitude stronger than linear optical activity and is well-pronounced above the multiphoton luminescence background. Because of its sensitivity, isotropic environment, and straightforward experimental geometry, HRS optical activity constitutes a fundamental experimental breakthrough in chiral photonics, for media including nano/metamaterials and chemical molecules. 


\section{Introduction}

The determination of chirality by optical means relies on measurable forms of optical activity (OA) (section~\ref{sec:background:Chirality:Chiroptics}). The chirality of a material can affect the absorption and phase velocity of circularly polarized light. By illuminating chiral molecules, consecutively with left and right-hand circularly polarized light, it is possible to measure, for instance, an asymmetry in absorption directly related to the molecule's chirality: Circular dichroism (CD), or circular intensity difference (CID). 

With the invention of the laser, access to high intensity light enabled the discovery of nonlinear optical counterparts of OA, including second harmonic generation optical activity (section~\ref{sec:background:NonlinearOptics:chirality}) and corresponding effects in sum frequency generation (SFG)~\cite{Giordmaine1965, Belkin2000, Fischer2000}. These techniques require high concentrations, and have not been widely adopted as routine chiroptical probes. A major drawback is that SFG intensities cannot distinguish between left and right-handed molecules~\cite{Valev2013b}. 
As for SHG, these processes require symmetry-breaking interfaces and are more suited to examining solids than liquids~\cite{Fischer2005a, Collins2017}. In solids however molecules can exhibit anisotropy that can easily mask the chiral response by contributing to linear dichroism, linear birefringence and circular birefringence~\cite{Kuroda2001}.
In solids, disentangling chirality from anisotropy effects is challenging, particularly for nonlinear optics~\cite{Hooper2017} (section~\ref{sec:results:OAinPlanarNanohelices}). More chiroptical processes have been reported in higher order nonlinearities, for instance, two-photon absorption circular dichroism~\cite{Tinoco1975, DeBoni2008, Toro2010} – a third order process. However, higher order nonlinearities are generally weaker, and the associated experimental techniques are complex. Therefore, nonlinear optical effects capable of distinguishing chiral forms in a liquid are highly desirable. 

Here, we demonstrate Optical Activity in Hyper-Rayleigh Scattering (HRS), 5 orders of magnitude more pronounced than linear OA. HRS~\cite{Clays1991b, Clays1992} occurs when incident light at a fundamental frequency is scattered at the second harmonic frequency, and is used to determine the symmetry of randomly oriented molecules in a liquid~\cite{Verbiest1994a}.
For our experiments, we made use of meta-molecules (chiral metal nanohelices) and demonstrate a strong HRS signal, well distinguishable above multiphoton luminescence. Polarization analyses of the HRS establishes that the nanohelices are isotopically arranged in the liquid. The nanohelices give rise to an HRS-CID signal that reverses for opposite chirality of the nanohelices and is much larger than the two- and three-photon luminescence CID. Moreover, the HRS and HRS-CID signals unambiguously follow variations in fundamental wavelength. Crucially, in comparison to linear OA, the HRS-CID only occurs near the focal point of light, which opens up a range of applications in tiny volumes of liquids. 
In HRS-CID, incident light at a frequency $\omega$ is left- circularly polarized (LCP) or right-circularly polarized (RCP). Meta-molecules in a liquid scatter the incident light into the second-harmonic frequency $2\omega$. This hyper-Rayleigh scattered light can be detected at $\SI{90}{\degree}$ to the incident beam propagation direction. Due to the chirality of the scatterers, HRS of different intensity is produced for LCP and RCP light.


\section{Results}

\begin{figure}[htb!]	
    \centering	
    \includegraphics[scale=1]{./figures/results/HRS/sample_schematic.pdf}
    \caption{\label{fig:results:HRS:sample_schematic}
    \textbf{a)} Dimensions (center-to-center) of the nanohelices isotropically dispersed in water. Helix height is $\SI{110}{\nano\m}$, loop diameter is $\SI{50}{\nano\m}$, loop pitch is $\SI{55}{\nano\m}$, and wire-diameter is $\SI{25}{\nano\m}$. \textbf{b)} Transmission Electron Microscopy (TEM) images of left- and right-handed helices. }	
\end{figure}

\begin{figure}[htb!]	
    \centering	
    \includegraphics[scale=1]{./figures/results/HRS/linear_data.pdf}
    \caption{\label{fig:results:HRS:linear_data}
    Linear characterization of the nanohelix solutions. Left: ellipticity spectra, as measured with a CD spectrometer, through a $\SI{1}{\centi\m}$ path length filled with left- and right-handed nanohelix suspensions. Right: corresponding normalized extinction spectra from the left- and right-handed nanohelix suspensions. Extinction is obtained from the transmission spectrum and describes both absorption and scattering losses.}	
\end{figure}

The meta-molecules are Ag ``nanohelices'', fabricated by H.-H. Jeong and P. Fischer using a glancing-angle shadow growth method~\cite{Gibbs2014}, suspended in water. The nanohelices' dimensions, presented in figure~\ref{fig:results:HRS:sample_schematic}a, are substantially smaller than the wavelength of illumination ($\SI{720}{\nano\m}$-$\SI{780}{\nano\m}$). Since the wire radius ($\SI{12.5}{\nano\m}$) is comparable to the skin depth of Ag for $\SI{760}{\nano\m}$ light~\cite{Johnson1972}, each helix acts as a continuous helical arrangement of effective dipoles. Transmission Electron Microscopy (TEM) images of both chiral forms (enantiomorphs) are shown in figure~\ref{fig:results:HRS:sample_schematic}b. 
After fabrication on Si wafers, the wafers are cut into $\SI{1}{\centi\meter\squared}$ pieces that are each sonicated into $\SI{1.4}{\milli\litre}$ water, with $\SI{1}{\milli\mole}$ of sodium citrate stabilizer to create $\approx 20$ picomolar suspensions. These suspensions are stable over several days and can be dispensed into standard glass cuvettes for characterization with linear and nonlinear optical techniques.  

Figure~\ref{fig:results:HRS:linear_data} shows linear ellipticity and extinction spectra (upper and lower panels respectively) for both enantiomorphs, obtained using a commercial Applied Photophysics Chirascan. A $\text{N}_2$-cooled Xe arc lamp is linearly polarized and spectrally separated in space with a pair of prisms, and a variable-width slit is used to select a wavelength with $\SI{0.1}{\nano\m}$ resolution. A photoelastic modulator (PEM) modulates the beam between LCP and RCP states, which is then directed through the sample cuvette and onto a photomultiplier tube (PMT) detector. The Chirascan then simultaneously measures total extinction and circular dichroism and constructs spectral data by scanning the wavelength.
The ellipticity $\theta$ is a measure of the CD:
\begin{equation}
    \theta (\text{deg.}) = \frac{180}{\pi} \tan\left( \frac{\sqrt{I_{RCP}} - \sqrt{I_{LCP}}}{\sqrt{I_{RCP}} + \sqrt{I_{LCP}}} \right) \approx \frac{180}{\pi} \Delta A \left( \frac{\ln 10}{4} \right)
\end{equation}
where $I_{RCP}$ and $I_{LCP}$ denote the intensity of RCP and LCP light, respectively, and $\Delta A = A_{LCP} - A_{RCP}$ is the difference in the absorbance of LCP and RCP light transmitted through the cuvette. 
The ellipticity spectra exhibit a characteristic bisignate signature that reverses with the handedness of the nanohelices. Their small asymmetry in peak maxima is due to a slight difference in concentration, attributable to experimental variation in the sonication process. Additionally, because the two chiral forms of the nanohelices are grown separately, small imperfections in the structural dimensions are also present, resulting in a slight shift in peak wavelength. The small effect of these imperfections can be seen from the extinction spectra, where the lines deviate only above $\SI{550}{\nano\m}$. The extinction is proportional to both the absorption and the scattering from the nanohelices; such scattering can also occur at the second harmonic frequency of illumination. 

\begin{figure}[htb!]	
    \centering	
    \includegraphics[scale=1]{./figures/results/HRS/experiment_schematic.pdf}
    \caption{\label{fig:results:HRS:experiment_schematic}
    Schematic diagram of the hyper-Rayleigh scattering circular dichroism experimental setup. QWP: quarter wave-plate; LP filter: long-pass filter; BP filter: band pass filter. }	
\end{figure}

Figure~\ref{fig:results:HRS:experiment_schematic} shows the experimental setup used to measure HRS-CD. 
$\SI{18.9}{\milli\watt}\pm\SI{0.1}{\milli\watt}$ of pulsed light ($\SI{139}{\kilo\watt}$ peak power, appendix~\ref{sec:appendix:hardware:maitai}) with a tunable center wavelength, was horizontally polarized (p-polarization), before passing through a quarter-wave plate mounted in an automatic rotation stage to give LCP or RCP light. An RG665 long pass filter removed any existing SHG from the beam, before a $\SI{200}{\nano\m}$ focal-length achromatic lens focused the incident light onto our cuvette. A $\SI{50}{\milli\m}$ focal-length collection lens positioned $\SI{50}{\milli\m}$ from the cuvette, along with a $\SI{50}{\milli\m}$ focal length curved mirror positioned $\SI{100}{\milli\m}$ from the opposite side of the cuvette, collected and collimated scattered light. A $\SI{200}{\milli\m}$ focal length lens then focused the collected light through one of several band-pass filters, onto a photomultiplier tube (PMT). The PMT output was pre-amplified and sent to an SRS SR400 Gated Photon Counter. 
By changing the band-pass filter, in $\SI{10}{\nano\m}$ increments, a spectrum of the multiphoton scattering is measured.

\begin{figure}[htb!]	
    \centering	
    \includegraphics[scale=1]{./figures/results/HRS/hrs_data.pdf}
    \caption{\label{fig:results:HRS:hrs_data}
    Multi-photon scattering spectra for left- and right-handed helices, under LCP and RCP illumination. Results obtained for fundamental wavelengths of $\SI{720}{\nano\m}$, $\SI{740}{\nano\m}$ and $\SI{780}{\nano\m}$ are shown in blue, green, and red, respectively. Vertical colored lines mark the HRS (second-harmonic) wavelength, demonstrating clear peaks above the multiphoton luminescence background. The HRS unambiguously follows variations of the fundamental. }	
\end{figure}

\begin{figure}[htb!]	
    \centering	
    \includegraphics[scale=1]{./figures/results/HRS/hrs_cd_data.pdf}
    \caption{\label{fig:results:HRS:hrs_cd_data}
    Circular difference ($I_{RCP}^{MS}-I_{LCP}^{MS}$) in multiphoton scattering intensity, for both left-handed (LH) and right-handed (RH) nanohelix suspensions. Again, pump wavelengths of $\SI{720}{\nano\m}$, $\SI{740}{\nano\m}$ and $\SI{780}{\nano\m}$ are shown in blue, green, and red, respectively, and vertical coloured lines mark the HRS (second-harmonic) wavelength. The HRS circular difference clearly reverses between nanohelix enantiomorphs.}	
\end{figure}

Figure~\ref{fig:results:HRS:hrs_data} shows the obtained multiphoton scattering spectra for both left- and right-handed nanohelix suspensions, under LCP and RCP illumination at three fundamental wavelengths: $\SI{720}{\nano\m}$, $\SI{740}{\nano\m}$ and $\SI{780}{\nano\m}$ (shown in blue, green, and red respectively). 
The second-harmonic is marked by a vertical line, with the shaded region denoting the bandwidth of the band-pass filter used. At all three fundamental wavelengths, a clear peak is observed at the second-harmonic, corresponding to hyper-Rayleigh scattering well above the multiphoton luminescence background. Importantly, a clear difference in HRS intensity between RCP and LCP illumination is also observed. 
This effect is emphasized in figure~\ref{fig:results:HRS:hrs_cd_data}, which shows the circular difference in multiphoton scattering intensities ($I_{RCP}^{MS}-I_{LCP}^{MS}$) for both enantiomorphs. At all three fundamental wavelengths, a clear peak in circular difference is observed at the second-harmonic, reversing sign between enantiomorphs indicating an intrinsically chiral origin. The HRS-CID is significantly larger than the neighboring two- and three-photon luminescence CID, which are also recorded. 

\begin{figure}[htb!]	
    \centering	
    \includegraphics[scale=1]{./figures/results/HRS/hrs_linpol_data.pdf}
    \caption{\label{fig:results:HRS:hrs_linpol_data}
    The nanohelices are isotropically suspended in water. \textbf{a)} Schematic of the setup used to measure hyper-Rayleigh scattering of linearly polarised incident light. \textbf{b)} P-polarised (blue crosses) and S-polarised (orange dots) HRS intensity at $\SI{360}{\nano\m}$, as the polarization of $\SI{720}{\nano\m}$ incident light is rotated. Both left-handed (LH) and right-handed (RH) nanohelix suspensions exhibit no variation outside of experimental uncertainty, demonstrating a clear isotropic arrangement of helices within the liquid suspension.}	
\end{figure}

To further verify the chiral origin of the measured HRS-CID, linearly polarized HRS measurements are performed. Figure~\ref{fig:results:HRS:hrs_linpol_data}a shows the setup used in these measurements. Here, the incident polarization is linear and can be freely rotated. An analyzing polarizer is placed before the detector, allowing the HRS signal to be decomposed into P-polarised (horizontal) and S-polarised (vertical) components. Both left- and right-handed nanohelix suspensions were examined, at an incident wavelength of $\SI{720}{\nano\m}$, with a $\SI{360}{\nano\m}$ band-pass filter at the output, selecting the HRS. 
Figure~\ref{fig:results:HRS:hrs_linpol_data}b shows the P- and S-polarised components of HRS intensity as the incident polarization is rotated. For both P- and S-polarized HRS, and for both enantiomorphs, it can be seen that HRS intensity remains constant. This result establishes an isotropic arrangement of nanohelices, and indicates a purely dipolar origin of the HRS response~\cite{Hao2002b, verbiest2009second}. 

\section{Discussion}

In hyper-Rayleigh scattering experiments, the measured second harmonic intensity originates from the hyper-polarizability of the (achiral) scatterers: $I(2\omega) \propto \langle \beta^{2}_{HRS} \rangle$.
In our case, in addition to the achiral hyper-polarizability ($\beta^{ach}_{HRS}$), there is a chiral contribution to the hyper-polarizability ($\beta^{ch}_{HRS}$), which changes sign depending on the direction of circularly polarized light. Consequently, the intensity at the second harmonic can be written as $I\left( {2\omega } \right) \propto \left\langle {{{\left( {\beta _{HRS}^{ach} \pm \beta _{HRS}^{ch}} \right)}^2}} \right\rangle$.

For the purposes of comparison to the linear CD spectra, we can characterize the HRS-CD by a polarization ellipticity:
\begin{equation}
    \theta^{HRS} (\text{deg.}) = \frac{180}{\pi} \tan\left( \frac{\sqrt{I_{RCP}^{HRS}} - \sqrt{I_{LCP}^{HRS}}}{\sqrt{I_{RCP}^{HRS}} + \sqrt{I_{LCP}^{HRS}}} \right).
\end{equation}
Here, $I_{RCP}^{HRS}$ and $I_{LCP}^{HRS}$ are the electric field intensities of light scattered at the second harmonic for RCP and LCP illumination, respectively. 
The measured linear and HRS ellipticities are given in table~\ref{table:HRS:ellipticity}. Note that in these data the linear- and HRS-CD are expected to have an opposite sign, due to experimental geometry.
Table~\ref{table:HRS:ellipticity} indicates that the measured HRS-CD effect is 3 to 4 times stronger than the linear CD across the range of wavelengths studied. However, when considering the volume of the liquid interacting with incident light, the HRS-CD effect appears significantly more sensitive than its linear counterpart. 
For a Gaussian beam, at a position $z$ from the focus, with a beam waist $w_0$ and a Rayleigh range $z(r)$, the beam radius $w(z)$ is given by $w(z) = {w_0}\sqrt {1 + (z/z_R)^2}$. 
We define the beam volume as the integral of the spot area $\pi w(z)^2$ between two positions $z_i$ (initial) and $z_f$ (final) as:
\begin{equation}
    V_{beam} = \int^{z_f}_{z_i} \pi \left( {w_0}^2 \sqrt {1 + (z/z_R)^2} \right)^2 dz.
\end{equation} 
For these experiments, we assume that significant HRS occurs within the Rayleigh range of the fundamental beam, and so the effective interaction volume is given by $V_{beam}$ integrated between $-z_R$ and $z_R$. 
Considering the $\SI{20}{\centi\m}$ focal length lens used in our experiments, with an incident beam of waist $\SI{1.2}{\milli\m}$ and $\lambda = \SI{740}{\nano\m}$ fundamental light, the waist at the focus ${w_f} = (\lambda f)/(\pi {w_0})$ is found to be $\approx \SI{40}{\micro\m}$, with a Rayleigh range of $\approx \SI{6}{\milli\m}$. From this, we assume that we measure HRS from the full $\SI{10}{\milli\m}$ length of cuvette. 
The effective beam volume ($V_{beam}$) is then $\approx \SI{10e-11}{\m\cubed}$ ($=\SI{10}{\nano\litre}$). 
By comparison, the commercial CD spectrometer uses a $\SI{10}{\milli\m}$ optical path length with an approximately constant beam radius of $\approx \SI{1}{\milli\m}$, giving an effective beam volume of  $\approx \SI{10e-6}{\m\cubed}$ ($=\SI{1}{\milli\litre}$). 
This value is 5 orders of magnitude larger compared to the nonlinear case. Consequently, the HRS-CD effect is significantly more sensitive than its linear counterpart. 

\begin{table}[tbp]
    \begin{tabular}{llll}
    \firsthline
                                                                & \textbf{\begin{tabular}[c]{@{}l@{}}Fundamental \\ wavelength (nm)\end{tabular}} & \textbf{\begin{tabular}[c]{@{}l@{}}Linear \\ ellipticity (mdeg.)\end{tabular}} & \textbf{\begin{tabular}[c]{@{}l@{}}HRS \\ ellipticity (mdeg.)\end{tabular}} \\ 
    \hline
    \multirow{3}{*}{\begin{tabular}[c]{@{}l@{}}Right-handed \\ helices\end{tabular}}  & 720                                                                    & 920                                                                   & -3299                                                              \\
                                                                & 740                                                                    & 803                                                                   & -3356                                                              \\
                                                                & 780                                                                    & 569                                                                   & -1890                                                              \\ 
    \hline
    \multirow{3}{*}{\begin{tabular}[c]{@{}l@{}}Left-handed \\ helices\end{tabular}}& 720                                                                    & -370                                                                  & 1811                                                               \\
                                                                & 740                                                                    & -293                                                                  & 1455                                                               \\
                                                                & 780                                                                    & -164                                                                  & 1334                                                               \\ 
    \lasthline
    \end{tabular}
    \caption{Linear and HRS ellipticities for left- and right-handed nanohelix suspensions. Values are given at the three fundamental wavelengths used in the HRS experiments.}
    \label{table:HRS:ellipticity}
\end{table}


\section{Conclusions}
Our data demonstrate that it is possible to perform HRS-CID characterization in liquid containers that are much smaller than the standard glass cuvettes. The range of available miniaturized cuvettes, narrow capillaries, microfluidic channels and hollow-core optical fibers position this new chiroptical characterization directly within the lab-on-a-chip paradigm. Importantly, HRS-CID occurs in isotropic liquids, freely from unwanted anisotropic contributions. Measuring pure chirality is key for optimizing the chirality parameters of metamaterials and for unlocking successful applications. Moreover, because it is well-established that the chiroptical properties of plasmonic nanoparticles can be leveraged to increase the optical response of molecules, our new technology will allow the miniaturized characterization of synthesized chemicals. For instance, in the case of many compounds, whose synthesis takes months and yields preciously few milligrams of material, chiral characterization needs to be performed at various stages squeezing the yields even further. Increased chiroptical sensitivity will dramatically reduce sampling volumes and will ultimately improve production. 